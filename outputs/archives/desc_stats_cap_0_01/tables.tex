%-----------------------------------------------------------------
% Tab 1
%-----------------------------------------------------------------
\begin{table}[H]
	\centering 
	\begin{adjustbox}{max width=\textheight}
		\begin{threeparttable}
			\caption{Summary statistics}
			\label{tab:summ_stats}
			\begin{tabular}{@{}l*{5}{c}@{}}
				\toprule
				Variable 		& 
				Obs. 			& 	 
				Mean			&
				St. Dev.		& 
				Min				&
				Max 			\\
				\midrule
				\multicolumn{6}{@{}l}{\textit{Panel A. 2014}}			\\	
				\primitiveinput{tables/summ_stats_2014.tex} 
				\midrule 
				\multicolumn{6}{@{}l}{\textit{Panel B. 2009}}			\\				
				\primitiveinput{tables/summ_stats_2009.tex}
				\midrule
				\multicolumn{6}{@{}l}{\textit{Panel C. 2005}}			\\			
				\primitiveinput{tables/summ_stats_2005.tex}	
				\midrule
				\multicolumn{6}{@{}l}{\textit{Panel D. Overall}}		\\			
				\primitiveinput{tables/summ_stats_pooled.tex} 			 				
				\bottomrule
			\end{tabular}
			\begin{tablenotes}
				\setlength\labelsep{0pt}
				\footnotesize
				\item \textit{Notes}: This table presents summary statistics for the 2005, 2009 and 2014 samples. Sex is a dummy variable that equals 1 if the individual is male and 0 if the individual is female. Area of residence is a dummy variable that equals 1 if the individual lives in urban area and 0 if the individuals lives in rural area. Household size measures the number of people in one household. 
			\end{tablenotes}
		\end{threeparttable}
	\end{adjustbox}
\end{table}
%-----------------------------------------------------------------
% Tab 2
%-----------------------------------------------------------------
\newpage 

\begin{table}[H]
	\centering 
	\begin{adjustbox}{width=\linewidth}
		\begin{threeparttable}
			\caption{Test for equality of means for key variables pre-treatment, 2009}
			\label{tab:balance}
			\begin{tabular}{@{}l*{6}{c}@{}}
				\toprule
								&
				\multicolumn{3}{c}{Before matching}	&
				\multicolumn{3}{c}{After matching}	\\
				\cmidrule(lr){2-4}	
				\cmidrule(l){5-7}
								&
				Not 			&
								&
								&
				Not				&
								&
								\\
								&
				Eligible		&	
				Eligible		& 
				Diff 			&
				Eligible		&
				Eligible		& 
				Diff 			\\
				Variables 		& 
				(1)				&
				(2)				&
				(3)				&
				(4)				&
				(5)				&
				(6)				\\
				\midrule 
				Sex&	0.61&	0.53&	0.08&	0.49&	0.53&	0.04\\
&	(0.01)&	(0.01)&	(0.02)&	(0.01)&	(0.01)&	(0.01)\\
Years of education&	6.08&	5.98&	0.10&	6.24&	6.05&	-0.19\\
&	(0.13)&	(0.08)&	(0.16)&	(0.09)&	(0.08)&	(0.12)\\
Area of residencey: urban&	0.67&	0.73&	-0.06&	0.74&	0.74&	-0.01\\
&	(0.01)&	(0.01)&	(0.02)&	(0.01)&	(0.01)&	(0.01)\\
Age&	41.32&	42.31&	-0.98&	42.56&	42.16&	-0.40\\
&	(0.40)&	(0.26)&	(0.48)&	(0.26)&	(0.26)&	(0.37)\\
Household size&	5.52&	5.23&	0.29&	5.28&	5.24&	-0.05\\
&	0.08&	0.05&	0.09&	0.05&	0.05&	0.07\\
 \\ [-3ex]
				\bottomrule
			\end{tabular}
			\begin{tablenotes}
				\setlength\labelsep{0pt}
				\footnotesize
				\item \textit{Notes}: This table reports descriptive statistics on the key variables. Standard errors are shown in parentheses. The estimations for column (1), (2), (4), and (5) come from a regression framework where the independent variable is the interpect of the variable. Column (3) and (6) present the differences between eligible and non-eligible individuals. 
			\end{tablenotes}
		\end{threeparttable}
	\end{adjustbox}
\end{table}
%-----------------------------------------------------------------
% Tab 3
%-----------------------------------------------------------------
\newpage 

\begin{table}[H]
	\centering 
	\begin{adjustbox}{width=\linewidth}
		\begin{threeparttable}
			\caption{ Parallel Trends - Using sample 2005--2009}
			\label{tab:summ_stats_pooled}
			\begin{tabular}{@{}l*{4}{c}@{}}
				\toprule
								&
				\multicolumn{4}{c}{Dependent variable: Log of Real Income} \\ 
				\cmidrule(l){2-5}
				Variables 		& 
				(1)				&
				(2)				&
				(3)				&
				(4)				\\
				\midrule 
				\primitiveinput{tables/parallel_trends.tex} \\
				\midrule
				Primary Activity Fixed Effects	& Yes & Yes	& Yes & Yes \\
				Controls						& No  & Yes	& Yes & Yes \\
				Regional Fixed Effects			& No  & No	& Yes & Yes	\\
				Occupation Fixed Effects		& No  & No  & No  &	Yes	\\		 				
				\bottomrule
			\end{tabular}
			\begin{tablenotes}
				\setlength\labelsep{0pt}
				\footnotesize
				\item \textit{Notes}: This table reports OLS estimates of a matched sample. The matching estimator is single nearest-neighbour within a caliper of 0.01 imposing common support. Standard errors clustered at time plus primarty activity level are shown in parentheses. We use only the 2005 and 2009 sample for the parallel trends exercise. The variable $\textrm{Post}$ is a dummy variable equal to 1 if year is 2009 and 0 otherwise. The unit of observation is an individual. The individual controls are sex, area of residence, years of education, age, household size, four regional fixed effects, eighteen primary activity fixed effects, and nine occupation fixed effects. Statistical significance at the 1, 5, 10\% levels are indicated by ***,**, and *, respectively.	
			\end{tablenotes}
		\end{threeparttable}
	\end{adjustbox}
\end{table}

%-----------------------------------------------------------------
% Tab 3
%-----------------------------------------------------------------
\begin{table}[H]
	\centering 
	\begin{adjustbox}{width=\linewidth}
		\begin{threeparttable}
			\caption{Falsification Test - Using only employed workers}
			\label{tab:falsification}
			\begin{tabular}{@{}l*{4}{c}@{}}
				\toprule
								&
				\multicolumn{4}{c}{Dependent variable: Log of Real Income} \\ 
				\cmidrule(l){2-5}
				Variables 		& 
				(1)				&
				(2)				&
				(3)				&
				(4)				\\
				\midrule 
				\primitiveinput{tables/falsification.tex} \\
				\midrule
				Primary Activity Fixed Effects	& Yes & Yes	& Yes & Yes \\
				Controls						& No  & Yes	& Yes & Yes \\
				Regional Fixed Effects			& No  & No	& Yes & Yes	\\
				Occupation Fixed Effects		& No  & No  & No  &	Yes	\\		 				
				\bottomrule
			\end{tabular}
			\begin{tablenotes}
				\setlength\labelsep{0pt}
				\footnotesize
				\item \textit{Notes}: This table reports OLS estimates of a matched sample of only employed workers. The matching estimator is single nearest-neighbour within a caliper of 0.01 imposing common support. Standard errors clustered at time plus primarty activity level are shown in parentheses. The unit of observation is an individual. The individual controls are sex, area of residence, years of education, age, household size, four regional fixed effects, eighteen primary activity fixed effects, and nine occupation fixed effects. Statistical significance at the 1, 5, 10\% levels are indicated by ***,**, and *, respectively.	
			\end{tablenotes}
		\end{threeparttable}
	\end{adjustbox}
\end{table}

%-----------------------------------------------------------------
% Tab 4
%-----------------------------------------------------------------
\newpage 

\begin{table}[H]
	\centering 
	\begin{adjustbox}{width=\linewidth}
		\begin{threeparttable}
			\caption{Impact of the program on real income}
			\label{tab:main_did_gender}
			\begin{tabular}{@{}l*{6}{c}@{}}
				\toprule
								&
				\multicolumn{6}{c}{Dependent variable: Log of Real Income} \\ 
				\cmidrule(l){2-7}
								& 
				\multicolumn{2}{c}{Overall}	& 
				\multicolumn{2}{c}{Females} & 
				\multicolumn{2}{c}{Males}	\\
				\cmidrule(lr){2-3}
				\cmidrule(lr){4-5}
				\cmidrule(l){6-7}		
				Variables 		& 
				(1)				&
				(2)				&
				(3)				&
				(4)				& 
				(5)				& 
				(6)				\\
				\midrule 
				\primitiveinput{tables/main_did_gender.tex} \\
				\midrule
				Primary Activity Fixed Effects	& Yes & Yes	& Yes & Yes & Yes & Yes \\
				Controls						& Yes & Yes	& Yes & Yes & Yes & Yes \\
				Regional Fixed Effects			& Yes & Yes	& Yes & Yes	& Yes & Yes \\
				Occupation Fixed Effects		& No  & Yes & No  &	Yes	& No  & Yes \\		 				
				\bottomrule
			\end{tabular}
			\begin{tablenotes}
				\setlength\labelsep{0pt}
				\footnotesize
				\item \textit{Notes}: This table reports OLS estimates of a matched sample. The matching estimator is single nearest-neighbour within a caliper of 0.01 imposing common support. Standard errors clustered at time plus primarty activity level are shown in parentheses. The unit of observation is an individual. The individual controls are sex, area of residence, years of education, age, household size, four regional fixed effects, eighteen primary activity fixed effects, and nine occupation fixed effects. Statistical significance at the 1, 5, 10\% levels are indicated by ***,**, and *, respectively.				
			\end{tablenotes}
		\end{threeparttable}
	\end{adjustbox}
\end{table}


%-----------------------------------------------------------------
% Tab 5
%-----------------------------------------------------------------
\newpage 

\begin{table}[H]
	\centering 
	\begin{adjustbox}{width=\linewidth}
		\begin{threeparttable}
			\caption{Heterogeneity by education}
			\label{tab:main_did_education}
			\begin{tabular}{@{}l*{9}{c}@{}}
				\toprule
								&
				\multicolumn{9}{c}{Dependent variable: Log of Real Income} \\ 
				\cmidrule(l){2-9}
								& 
				\multicolumn{3}{c}{Primary}		& 
				\multicolumn{3}{c}{High-school} & 
				\multicolumn{3}{c}{Above high-school}	\\
				\cmidrule(lr){2-4}
				\cmidrule(lr){5-7}
				\cmidrule(l){8-10}	
								&
				Overall 		& 
				Females 		& 
				Males			& 
				Overall 		& 
				Females 		& 
				Males			& 
				Overall 		& 
				Females 		& 
				Males			\\								
				Variables 		& 
				(1)				&
				(2)				&
				(3)				&
				(4)				& 
				(5)				& 
				(6)				& 
				(7)				& 
				(8)				& 
				(9)				\\
				\midrule 
				\primitiveinput{tables/main_did_educ.tex} \\
				\midrule
				Primary Activity Fixed Effects	& Yes & Yes	& Yes & Yes & Yes & Yes & Yes & Yes & Yes \\
				Controls						& Yes & Yes	& Yes & Yes & Yes & Yes & Yes & Yes & Yes \\
				Regional Fixed Effects			& Yes & Yes	& Yes & Yes	& Yes & Yes & Yes & Yes & Yes \\
				Occupation Fixed Effects		& Yes & Yes & Yes &	Yes	& Yes & Yes & Yes & Yes & Yes \\
				\bottomrule
			\end{tabular}
			\begin{tablenotes}
				\setlength\labelsep{0pt}
				\footnotesize
				\item \textit{Notes}: This table reports OLS estimates of a matched sample. The matching estimator is single nearest-neighbour within a caliper of 0.01 imposing common support. Standard errors clustered at time plus primarty activity level are shown in parentheses. The unit of observation is an individual. The individual controls are sex, area of residence, years of education, age, household size, four regional fixed effects, eighteen primary activity fixed effects, and nine occupation fixed effects. Statistical significance at the 1, 5, 10\% levels are indicated by ***,**, and *, respectively.	
			\end{tablenotes}
		\end{threeparttable}
	\end{adjustbox}
\end{table}
%-----------------------------------------------------------------
% Tab 7
%-----------------------------------------------------------------
\begin{table}[H]
	\centering 
	\begin{adjustbox}{max width=\textheight}
		\begin{threeparttable}
			\caption{Heterogeneity by sectors}
			\label{tab:sectors}
			\begin{tabular}{@{}l*{6}{c}@{}}
				\toprule
								&
				\multicolumn{6}{c}{Dependent variable:} \\
								& 
				\multicolumn{6}{c}{Log of Real Income}	\\
				\cmidrule(l){2-7}
								&
				\multicolumn{2}{c}{Overall} & 
				\multicolumn{2}{c}{Females} & 
				\multicolumn{2}{c}{Males}	\\
				\cmidrule(lr){2-3}
				\cmidrule(lr){4-5}
				\cmidrule(l){6-7}
				Variables 		& 
				(1)				&
				(2)				&
				(3)				&
				(4)				&
				(5)				&
				(6)				\\
				\midrule 
				\primitiveinput{tables/main_did_sector.tex} \\
				\midrule
				Controls						& No   & Yes	& No   & Yes & No   & Yes \\
				Primary Activity Fixed Effects	& Yes  & Yes	& Yes  & Yes & Yes  & Yes \\
				Regional Fixed Effects			& Yes  & Yes	& Yes  & Yes & Yes  & Yes \\
				Occupation Fixed Effects		& Yes  & Yes  	& Yes  & Yes & Yes  & Yes \\		 				
				\bottomrule
			\end{tabular}
			\begin{tablenotes}
				\setlength\labelsep{0pt}
				\footnotesize
				\item \textit{Notes}: This table reports OLS estimates of a matched sample. The matching estimator is single nearest-neighbour within a caliper of 0.01 imposing common support. Standard errors clustered at time plus primarty activity level are shown in parentheses. The unit of observation is an individual. The individual controls are sex, area of residence, years of education, age, household size, four regional fixed effects, eighteen primary activity fixed effects, and nine occupation fixed effects. Statistical significance at the 1, 5, 10\% levels are indicated by ***,**, and *, respectively.	
			\end{tablenotes}
		\end{threeparttable}
	\end{adjustbox}
\end{table}


%-----------------------------------------------------------------
% Tab 8
%-----------------------------------------------------------------
\newpage 

\begin{table}[H]
	\centering 
	\begin{adjustbox}{width=\linewidth}
		\begin{threeparttable}
			\caption{Impact of the program on real income (winsorizing)}
			\label{tab:main_did_gender_winsor}
			\begin{tabular}{@{}l*{6}{c}@{}}
				\toprule
								&
				\multicolumn{6}{c}{Dependent variable: Log of Real Income} \\ 
				\cmidrule(l){2-7}
								& 
				\multicolumn{3}{c}{1th and 99th}	& 
				\multicolumn{3}{c}{10th and 90th} 	\\
								& 
				Overall			& 
				Females			&
				Males			& 
				Overall			& 
				Females			& 
				Males			\\								
				\cmidrule(lr){2-4}
				\cmidrule(lr){5-7}	
				Variables 		& 
				(1)				&
				(2)				&
				(3)				&
				(4)				& 
				(5)				& 
				(6)				\\
				\midrule 
				\primitiveinput{tables/main_did_gender_win.tex} \\
				\midrule
				Primary Activity Fixed Effects	& Yes & Yes	& Yes & Yes & Yes & Yes \\
				Controls						& Yes & Yes	& Yes & Yes & Yes & Yes \\
				Regional Fixed Effects			& Yes & Yes	& Yes & Yes	& Yes & Yes \\
				Occupation Fixed Effects		& Yes & Yes & Yes &	Yes	& Yes & Yes \\		 				
				\bottomrule
			\end{tabular}
			\begin{tablenotes}
				\setlength\labelsep{0pt}
				\footnotesize
				\item \textit{Notes}: This table reports OLS estimates of a matched sample. The matching estimator is single nearest-neighbour within a caliper of 0.01 imposing common support. Standard errors clustered at time plus primarty activity level are shown in parentheses. The unit of observation is an individual. The individual controls are sex, area of residence, years of education, age, household size, four regional fixed effects, eighteen primary activity fixed effects, and nine occupation fixed effects. Statistical significance at the 1, 5, 10\% levels are indicated by ***,**, and *, respectively.	
			\end{tablenotes}
		\end{threeparttable}
	\end{adjustbox}
\end{table}

%-----------------------------------------------------------------
% Tab 9
%-----------------------------------------------------------------
\begin{table}[H]
	\centering 
	\begin{adjustbox}{max width=\textheight}
		\begin{threeparttable}
			\caption{Impact of the program on other outcomes}
			\label{tab:other_outcomes}
			\begin{tabular}{@{}l*{4}{c}@{}}
				\toprule
								&
				\multicolumn{4}{c}{Dependent variable:} \\ 
								&
				Received		&
				Weekly hours	&
				Months			& 
				Probability of	\\
								&
				training		&
				worked			&
				worked			&
				having a 2nd job \\				
				\cmidrule(l){2-5}
				Variables 		& 
				(1)				&
				(2)				&
				(3)				&
				(4)				\\
				\midrule 
				\multicolumn{5}{@{}l}{\textit{Panel A. Overall}}		\\				
				\primitiveinput{tables/main_did_other_outcomes.tex} 	\\
				\midrule
				\multicolumn{5}{@{}l}{\textit{Panel A. Overall}}		\\	
				\primitiveinput{tables/main_did_other_outcomes_females.tex} \\
				\midrule
				\multicolumn{5}{@{}l}{\textit{Panel C. Males}}			\\
				\primitiveinput{tables/main_did_other_outcomes_males.tex} \\
				\midrule			
				Primary Activity Fixed Effects	& Yes  & Yes	& Yes  & Yes \\
				Controls						& Yes  & Yes	& Yes  & Yes \\
				Regional Fixed Effects			& Yes  & Yes	& Yes  & Yes	\\
				Occupation Fixed Effects		& Yes  & Yes    & Yes  &	Yes	\\		 				
				\bottomrule
			\end{tabular}
			\begin{tablenotes}
				\setlength\labelsep{0pt}
				\footnotesize
				\item \textit{Notes}: The table reports OLS estimates. Standard errors clustered at the time plus primarty activity level are shown in parentheses. The unit of observation is an individual. The individual controls are sex, area of residence, years of education, age, household size, four regional fixed effects, eighteen primary activity fixed effects, and nine occupation fixed effects. Statistical significance at the 1, 5, 10\% levels are indicated by ***,**, and *, respectively.
			\end{tablenotes}
		\end{threeparttable}
	\end{adjustbox}
\end{table}
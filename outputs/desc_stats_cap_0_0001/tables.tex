%-----------------------------------------------------------------
% Tab 1
%-----------------------------------------------------------------
\begin{table}[H]
	\centering 
	\begin{adjustbox}{max width=\textheight}
		\begin{threeparttable}
			\caption{Summary statistics for self-employed workers in the LSMS}
			\label{tab:summ_stats}
			\begin{tabular}{@{}l*{5}{c}@{}}
				\toprule
				Variable 		& 
				Obs. 			& 	 
				Mean			&
				St. Dev.		& 
				Min				&
				Max 			\\
				\midrule
				\multicolumn{6}{@{}l}{\textit{Panel A. Overall (all years)}}		\\			
				\primitiveinput{tables/summ_stats_pooled.tex} 			
				\midrule						
				\multicolumn{6}{@{}l}{\textit{Panel B. 2005}}			\\			
				\primitiveinput{tables/summ_stats_2005.tex}	
				\midrule				
				\multicolumn{6}{@{}l}{\textit{Panel C. 2009}}			\\				
				\primitiveinput{tables/summ_stats_2009.tex}				
				\midrule		 				
				\multicolumn{6}{@{}l}{\textit{Panel D. 2014}}			\\	
				\primitiveinput{tables/summ_stats_2014.tex} 				
				\bottomrule
			\end{tabular}
			\begin{tablenotes}
				\setlength\labelsep{0pt}
				\footnotesize
				\item \textit{Notes}: This table presents summary statistics for the overall sample and each LSMS wave sample (2005, 2009, and 2014). Gender is a dummy variable that equals 1 if the individual is male and 0 otherwise. Area of residency is a dummy variable that equals 1 if the individual lives in an urban area and 0 otherwise. Real income was calculated by the 2006 consumer price index (CPI) from the Central Bank of Nicaragua. 
			\end{tablenotes}
		\end{threeparttable}
	\end{adjustbox}
\end{table}
%-----------------------------------------------------------------
% Tab 2
%-----------------------------------------------------------------
\newpage 

\begin{table}[H]
	\centering
	\begin{threeparttable}
	\caption{Test for equality of means for the pre-intervention variables (2009 LSMS)}
	\centering
	\begin{tabular}[t]{@{}l@{}l}
	\toprule
	\begin{tabular}[t]{@{}lccc}
				& \multicolumn{3}{c}{Before matching} 	\\ \cmidrule(lr){2-4}
				&			& Not		&		\\
				& Eligible  & Eligible 	& Diff	\\
	Variables	& (1) & (2) & (3)		\\
	\midrule
	\multicolumn{4}{@{}l}{\textit{Panel A. Covariates}}					\\
	\primitiveinput{tables/balance_cov.tex}
	\midrule
	\addlinespace 
	\multicolumn{4}{@{}l}{\textit{Panel B. Pre-intervention outcomes}}	\\	
	\primitiveinput{tables/balance_out.tex}
	\end{tabular}
	&
	\begin{tabular}[t]{Hccc}
				& \multicolumn{3}{c}{After matching}	\\ \cmidrule(l){2-4}
				&			& Not		&		\\
				& Eligible  & Eligible 	& Diff	\\
				& (4) & (5) & (6)		\\		
	\midrule
				&			&			&		\\	
	\primitiveinput{tables/balance_cov_matched.tex}
	\midrule
	\addlinespace 	
				&			&			&		\\	
	\primitiveinput{tables/balance_out_matched.tex}
	\end{tabular}
	\tabularnewline \bottomrule
	\end{tabular}
		\begin{tablenotes}
		\setlength\labelsep{0pt}
		\footnotesize
		\item \textit{Notes}: This table presents descriptive statistics on the pre-intervention variables from 2009 LSMS. Column (3) and (6) present the differences between the eligible and non-eligible groups. Robust standard errors are in parentheses. *** p<0.01, ** p<0.05, * p<0.1.
		\end{tablenotes}
	\end{threeparttable}
\end{table}
%-----------------------------------------------------------------
% Tab 3
%-----------------------------------------------------------------
\newpage 

\begin{table}[H]
	\centering 
		\begin{threeparttable}
			\caption{Test for parallel trends between 2005 and 2009}
			\label{tab:summ_stats_pooled}
			\begin{tabular}{@{}l*{5}{c}@{}}
				\toprule
								&
				\multicolumn{5}{c}{Dependent variable: Log of Real Income} \\ 
				\cmidrule(l){2-6}
				Variables 		& 
				(1)				&
				(2)				&
				(3)				&
				(4)				& 
				(5)				\\
				\midrule 
				\primitiveinput{tables/parallel_trends.tex} \\
				\midrule
				Controls						& No  	& Yes 	& Yes 	& Yes 	& Yes \\
				Regional Fixed Effects			& No 	& No	& Yes	& Yes	& Yes \\
				Occupation Fixed Effects		& No  	& No 	& No 	& Yes 	& Yes \\					
				Primary Activity Fixed Effects	& No  	& No 	& No 	& No 	& Yes \\ 					
				\bottomrule
			\end{tabular}
			\begin{tablenotes}
				\setlength\labelsep{0pt}
				\footnotesize
				\item \textit{Notes}: This table reports OLS estimates of a matched sample. The matching estimator is single nearest-neighbour within a caliper of 0.0001 imposing common support. We use only the 2005 and 2009 sample for test for parallel trends. Post is a dummy variable equal to 1 if year is 2009 and 0 if year is 2005. Eligibility is a dummy variable equal to 1 if individuals are eligible for the SBFE program and 0 otherwise. Controls include gender, age, household size, years of education, and area of residence, four regional fixed effects, nine occupation fixed effects, and eighteen primary activity fixed effects. Robust standard errors clustered at the year times eighteen primary economic sectors are shown in parentheses. *** p<0.01, ** p<0.05, * p<0.1.
			\end{tablenotes}
		\end{threeparttable}
\end{table}
%-----------------------------------------------------------------
% Tab 4
%-----------------------------------------------------------------
\begin{landscape}
\begin{table}[H]
	\centering 
	\begin{adjustbox}{width=\linewidth}
		\begin{threeparttable}
			\caption{Impact of SBFE program on income}
			\label{tab:main_did_gender}
			\begin{tabular}{@{}l*{9}{c}@{}}
				\toprule
								&
				\multicolumn{9}{c}{Dependent variable: Log of Real Income} \\ 
				\cmidrule(l){2-10}
								& 
				\multicolumn{3}{c}{Overall}	& 
				\multicolumn{3}{c}{Females} & 
				\multicolumn{3}{c}{Males}	\\
				\cmidrule(lr){2-4}
				\cmidrule(lr){5-7}
				\cmidrule(l){8-10}	
				Variables 		& 
				(1)				&
				(2)				&
				(3)				&
				(4)				& 
				(5)				& 
				(6)				& 
				(7)				& 
				(8)				& 
				(9)				\\
				\midrule 
				\primitiveinput{tables/main_did_gender.tex} \\
				\midrule
				Controls						& No  	& Yes 	& Yes 	& No  & Yes & Yes & No  & Yes 	& Yes\\
				Regional Fixed Effects			& No 	& Yes	& Yes	& No  & Yes & Yes & No  & Yes 	& Yes\\
				Occupation Fixed Effects		& No  	& No 	& Yes 	& No  & No  & Yes & No  & No 	& Yes\\
				Primary Activity Fixed Effects	& No  	& No 	& Yes 	& No  & No  & Yes & No  & No 	& Yes\\				 
				\bottomrule
			\end{tabular}
			\begin{tablenotes}
				\setlength\labelsep{0pt}
				\footnotesize
				\item \textit{Notes}: This table reports OLS estimates of a matched sample. The matching estimator is single nearest-neighbour within a caliper of 0.0001 imposing common support. Post is a dummy variable equal to 1 if year is 2014 and 0 if year is 2009. Eligibility is a dummy variable equal to 1 if individuals are eligible for the SBFE program and 0 otherwise. Controls include gender, age, household size, years of education, and area of residence, four regional fixed effects, nine occupation fixed effects, and eighteen primary activity fixed effects. Robust standard errors clustered at the year times eighteen primary economic sectors are shown in parentheses. *** p<0.01, ** p<0.05, * p<0.1.			
			\end{tablenotes}
		\end{threeparttable}
	\end{adjustbox}
\end{table}

%-----------------------------------------------------------------
% Tab 5
%-----------------------------------------------------------------
\newpage 

\begin{table}[H]
	\centering 
	\begin{adjustbox}{width=\linewidth}
		\begin{threeparttable}
			\caption{Heterogeneous treatment effects by education attainment}
			\label{tab:main_did_education}
			\begin{tabular}{@{}l*{9}{c}@{}}
				\toprule
								&
				\multicolumn{9}{c}{Dependent variable: Log of Real Income} \\ 
				\cmidrule(l){2-10}
								& 
				\multicolumn{3}{c}{Primary completed}		& 
				\multicolumn{3}{c}{Secondary completed} 	& 
				\multicolumn{3}{c}{Above secondary}			\\
								&
				\multicolumn{3}{c}{or less}					& 
				\multicolumn{3}{c}{or less} 				& 
				\multicolumn{3}{c}{school}					\\				
				\cmidrule(lr){2-4}
				\cmidrule(lr){5-7}
				\cmidrule(l){8-10}	
								&
				Overall 		& 
				Females 		& 
				Males			& 
				Overall 		& 
				Females 		& 
				Males			& 
				Overall 		& 
				Females 		& 
				Males			\\								
				Variables 		& 
				(1)				&
				(2)				&
				(3)				&
				(4)				& 
				(5)				& 
				(6)				& 
				(7)				& 
				(8)				& 
				(9)				\\
				\midrule 
				\primitiveinput{tables/main_did_educ.tex} \\
				\midrule
				Controls						& Yes  	& Yes 	& Yes 	& Yes  & Yes  & Yes & Yes  & Yes 	& Yes\\
				Regional Fixed Effects			& Yes 	& Yes	& Yes	& Yes  & Yes  & Yes & Yes  & Yes 	& Yes\\
				Occupation Fixed Effects		& Yes  	& Yes 	& Yes 	& Yes  & Yes  & Yes & Yes  & Yes 	& Yes\\
				Primary Activity Fixed Effects	& Yes  	& Yes 	& Yes 	& Yes  & Yes  & Yes & Yes  & Yes 	& Yes\\ 
				\bottomrule
			\end{tabular}
			\begin{tablenotes}
				\setlength\labelsep{0pt}
				\footnotesize
				\item \textit{Notes}: This table reports OLS estimates of a matched sample. The matching estimator is single nearest-neighbour within a caliper of 0.0001 imposing common support. Post is a dummy variable equal to 1 if year is 2014 and 0 if year is 2009. Eligibility is a dummy variable equal to 1 if individuals are eligible for the SBFE program and 0 otherwise. Controls include gender, age, household size, years of education, and area of residence, four regional fixed effects, nine occupation fixed effects, and eighteen primary activity fixed effects. Robust standard errors clustered at the year times eighteen primary economic sectors are shown in parentheses. *** p<0.01, ** p<0.05, * p<0.1.
			\end{tablenotes}
		\end{threeparttable}
	\end{adjustbox}
\end{table}
\end{landscape}
%-----------------------------------------------------------------
% Tab 6
%-----------------------------------------------------------------
\begin{table}[H]
	\centering 
	\begin{threeparttable}
		\caption{Falsification test on paid-employed workers between 2009 and 2014}
		\label{tab:falsification}
		\begin{tabular}{@{}l*{5}{c}@{}}
			\toprule
							&
			\multicolumn{5}{c}{Dependent variable: Log of Real Income} \\ 
			\cmidrule(l){2-6}
			Variables 		& 
			(1)				&
			(2)				&
			(3)				&
			(4)				& 
			(5)				\\
			\midrule 
			\primitiveinput{tables/falsification.tex} \\
			\midrule
			Controls						& No  	& Yes 	& Yes 	& Yes 	& Yes \\
			Regional Fixed Effects			& No 	& No	& Yes	& Yes	& Yes \\
			Occupation Fixed Effects		& No  	& No 	& No 	& Yes 	& Yes \\					
			Primary Activity Fixed Effects	& No  	& No 	& No 	& No 	& Yes \\ 				
			\bottomrule
		\end{tabular}
		\begin{tablenotes}
			\setlength\labelsep{0pt}
			\footnotesize
			\item \textit{Notes}: This table reports OLS estimates of a matched sample. The matching estimator is single nearest-neighbour within a caliper of 0.0001 imposing common support. We use the sample of paid- employed workers from 2009 and 2014 LSMS for the falsification test. Post is a dummy variable equal to 1 if year is 2014 and 0 if year is 2009. Eligibility is a dummy variable equal to 1 if individuals are eligible for the SBFE program and 0 otherwise. Controls include gender, age, household size, years of education, and area of residence, four regional fixed effects, nine occupation fixed effects, and eighteen primary activity fixed effects. Robust standard errors clustered at the year times eighteen primary economic sectors are shown in parentheses. *** p<0.01, ** p<0.05, * p<0.1.
		\end{tablenotes}
	\end{threeparttable}
\end{table}
%-----------------------------------------------------------------
% Tab 7
%-----------------------------------------------------------------
\begin{table}[H]
	\centering 
	\begin{adjustbox}{max width=\textheight}
		\begin{threeparttable}
			\caption{Possible mechanisms for the SBFE program impact}
			\label{tab:other_outcomes}
			\begin{tabular}{@{}l*{4}{c}@{}}
				\toprule
								&
				\multicolumn{4}{c}{Dependent variable:} \\ 
								&
				Received		&
				Weekly hours	&
				Months			& 
				Probability of	\\
								&
				training		&
				worked			&
				worked			&
				having a 2nd job \\				
				\cmidrule(l){2-5}
				Variables 		& 
				(1)				&
				(2)				&
				(3)				&
				(4)				\\
				\midrule 
				\multicolumn{5}{@{}l}{\textit{Panel A. Overall}}				\\				
				\primitiveinput{tables/main_did_other_outcomes.tex} 			\\ [-1em]
				\midrule
				\multicolumn{5}{@{}l}{\textit{Panel B. Females}}				\\	
				\primitiveinput{tables/main_did_other_outcomes_females.tex} 	\\ [-1em]
				\midrule
				\multicolumn{5}{@{}l}{\textit{Panel C. Males}}					\\
				\primitiveinput{tables/main_did_other_outcomes_males.tex} 		\\ [-1em]
				\midrule			
				Controls						& Yes & Yes	& Yes & Yes \\
				Regional Fixed Effects			& Yes & Yes	& Yes & Yes \\
				Occupation Fixed Effects		& Yes & Yes & Yes &	Yes	\\
				Primary Activity Fixed Effects	& Yes & Yes	& Yes & Yes	\\						 				
				\bottomrule
			\end{tabular}
			\begin{tablenotes}
				\setlength\labelsep{0pt}
				\footnotesize
				\item \textit{Notes}: This table reports OLS estimates of a matched sample. The matching estimator is single nearest-neighbour within a caliper of 0.0001 imposing common support. We use the sample of self-employed workers from 2009 and 2014 LSMS. Post is a dummy variable equal to 1 if year is 2014 and 0 if year is 2009. Eligibility is a dummy variable equal to 1 if individuals are eligible for the SBFE program and 0 otherwise. Received training is a dummy variable equal to 1 if individuals answered that they received training and 0 otherwise. Probability of having a 2nd job is a dummy variable equal to 1 if individuals answered that they have a second job and 0 otherwise. Controls include gender, age, household size, years of education, and area of residence, four regional fixed effects, nine occupation fixed effects, and eighteen primary activity fixed effects. Robust standard errors clustered at the year times eighteen primary economic sectors are shown in parentheses. *** p<0.01, ** p<0.05, * p<0.1.
			\end{tablenotes}
		\end{threeparttable}
	\end{adjustbox}
\end{table}
